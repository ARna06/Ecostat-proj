\section{Hypothesis Statements}

\section{Null and Alternative Hypotheses}

Let $IMR$ denote the infant mortality outcome (binary at child level, or rate per 1000 live births at aggregate level), $W$ the household wealth index (continuous or quintile), $A$ the mother's age (years or age group), $R$ the residence type (urban/rural), $E$ the mother's education level (years of schooling or categorical), and $C$ the fertility (number of children ever born, or average children per woman). Subscripts $MZ$ and $HP$ denote Mizoram and Himachal Pradesh respectively.

\subsection*{Test 1: Infant mortality vs Wealth index}

\paragraph{Model / Test:} Regress $IMR$ on $W$ (logistic regression for child-level binary IMR, or linear regression for aggregate IMR):
where $\mathbf{X}$ are controls (e.g., mother's age, education, residence).

\begin{itemize}
  \item \textbf{Mizoram}
    \begin{itemize}
      \item $H_{0}^{(MZ)}:\ \beta_W^{(MZ)} = 0$ (wealth index has no association with infant mortality in Mizoram).
      \item $H_{A}^{(MZ)}:\ \beta_W^{(MZ)} < 0$ (higher wealth is associated with lower infant mortality in Mizoram).
    \end{itemize}
  \item \textbf{Himachal Pradesh}
    \begin{itemize}
      \item $H_{0}^{(HP)}:\ \beta_W^{(HP)} = 0$.
      \item $H_{A}^{(HP)}:\ \beta_W^{(HP)} < 0$.
    \end{itemize}
\end{itemize}

\noindent\textit{Rationale:} Household wealth improves access to nutrition, healthcare and sanitation; hence a negative association is expected. 

\subsection*{Test 2: Infant mortality vs Mother's age}

\paragraph{Model / Test:} Compare IMR across maternal age categories (e.g., $<20$, $20$--$34$, $\ge 35$). Use logistic regression with age dummies or two-sample tests for proportions.

\begin{itemize}
  \item \textbf{Mizoram}
    \begin{itemize}
      \item $H_{0}^{(MZ)}:$ IMR is the same across maternal age groups (no difference between $<20$ and $20$--$34$).
      \item $H_{A}^{(MZ)}:$ IMR among mothers $<20$ is greater than IMR among mothers aged $20$--$34$.
    \end{itemize}
  \item \textbf{Himachal Pradesh}
    \begin{itemize}
      \item $H_{0}^{(HP)}:$ IMR is the same across maternal age groups.
      \item $H_{A}^{(HP)}:$ IMR among mothers $<20$ is greater than IMR among mothers aged $20$--$34$.
    \end{itemize}
\end{itemize}

\noindent\textit{Rationale:} Very young mothers face higher biological and socio-economic risks (e.g., poorer antenatal care), so infants of teen mothers often show worse outcomes. Test can be one-sided if theory is directional; include controls (education, wealth).

\subsection*{Test 3: Residence type vs Infant mortality}

\paragraph{Model / Test:} Compare proportions of infant deaths between rural and urban (Chi-square test of independence or logistic regression with residence dummy).

\begin{itemize}
  \item \textbf{Mizoram}
    \begin{itemize}
      \item $H_{0}^{(MZ)}:$ $p_{\text{rural}}^{(MZ)} = p_{\text{urban}}^{(MZ)}$ (no difference in infant mortality proportions).
      \item $H_{A}^{(MZ)}:$ $p_{\text{rural}}^{(MZ)} > p_{\text{urban}}^{(MZ)}$ (rural IMR higher than urban).
    \end{itemize}
  \item \textbf{Himachal Pradesh}
    \begin{itemize}
      \item $H_{0}^{(HP)}:$ $p_{\text{rural}}^{(HP)} = p_{\text{urban}}^{(HP)}$.
      \item $H_{A}^{(HP)}:$ $p_{\text{rural}}^{(HP)} > p_{\text{urban}}^{(HP)}$.
    \end{itemize}
\end{itemize}

\noindent\textit{Rationale:} Rural areas typically have lower health service access and worse infrastructure; therefore rural IMR is expected to be higher. For mountainous states, "rural" may include very remote habitations—interpretation should consider accessibility measures.

\subsection*{Test 4: Mother's Education vs Sanitation Access}

Let:
\begin{itemize}
    \item $E_i$ = Mother's education level for household $i$ (measured in completed years of schooling or categorical levels such as no schooling, primary, secondary, higher).
    \item $S_i$ = Sanitation access for household $i$, where 
    \[
    S_i = 
    \begin{cases}
    1, & \text{if household has access to improved sanitation facilities (toilet with water, septic tank, etc.)} \\
    0, & \text{otherwise.}
    \end{cases}
    \]
\end{itemize}

We test whether an increase in maternal education ($E_i$) is associated with an increase in the probability of having improved sanitation access ($S_i$).

%----------------------------------------------
\begin{itemize}
  \item \textbf{Mizoram}
  \begin{itemize}
    \item $
H_{0}^{MZ}: \beta_{E}^{MZ} = 0 \quad \Rightarrow \quad E_i \text{ has no significant effect on sanitation access } (S_i) \text{ in Mizoram.}
$
    \item $
H_{1}^{MZ}: \beta_{E}^{MZ} > 0 \quad \Rightarrow \quad \text{Higher } E_i \text{ increases the likelihood of improved sanitation } S_i \text{ in Mizoram.}
$
  \end{itemize}
  \item \textbf{Himachal Pradesh}
  \begin{itemize}
    \item $
H_{0}^{HP}: \beta_{E}^{HP} = 0 \quad \Rightarrow \quad E_i \text{ has no significant effect on sanitation access } (S_i) \text{ in Himachal Pradesh.}
$
    \item $
H_{1}^{HP}: \beta_{E}^{HP} > 0 \quad \Rightarrow \quad \text{Higher } E_i \text{ increases the likelihood of improved sanitation } S_i \text{ in Himachal Pradesh.}
$
  \end{itemize}
\end{itemize}   


%----------------------------------------------
\textit{Rationale:}  
Educated mothers are more aware of hygiene, disease prevention, and government sanitation schemes (e.g., Swachh Bharat Mission). Education also improves socio-economic status and decision-making power within households. Therefore, higher maternal education is expected to positively influence access to and use of improved sanitation facilities.


\subsection*{Test 5: Average number of children born vs Women's education}

\paragraph{Model / Test:} Test for a negative association between women's education level $E$ and fertility $C$ (using ANOVA across education categories).

\begin{itemize}
  \item \textbf{Mizoram}
    \begin{itemize}
      \item $H_{0}^{(MZ)}:$ $\mathbb{E}[C|E=\text{low}] = \mathbb{E}[C|E=\text{medium}] = \mathbb{E}[C|E=\text{high}]$ (no difference in mean children across education levels).
      \item $H_{A}^{(MZ)}:$ $\mathbb{E}[C|E=\text{low}] > \mathbb{E}[C|E=\text{medium}] > \mathbb{E}[C|E=\text{high}]$ (higher education associated with fewer children; i.e., a negative trend).
    \end{itemize}
  \item \textbf{Himachal Pradesh}
    \begin{itemize}
      \item $H_{0}^{(HP)}:$ $\mathbb{E}[C|E=\text{levels}]$ equal across education levels.
      \item $H_{A}^{(HP)}:$ $\mathbb{E}[C|E]$ declines with increasing education (negative association / monotonic trend).
    \end{itemize}
\end{itemize}

\noindent\textit{Rationale:} Education increases women's labour market opportunities and contraceptive knowledge/use, and tends to delay marriage and first birth — all factors that lower completed fertility.

\subsection*{Notes on testing strategy and controls}

\begin{itemize}
  \item Prefer regression frameworks (logistic for binary child-level IMR, count models for fertility) with robust standard errors and relevant controls (wealth, residence, maternal age, parity, access to health facilities) to estimate partial associations.
  \item Where hypotheses are directional (expecting a negative association), one-sided tests may be used — but two-sided tests are more conservative and commonly reported.
  \item For group comparisons (age groups, education categories), supplement regression estimates with simple descriptive statistics (means, proportions) and visualisations (boxplots, bar charts).
  \item Always report effect sizes (odds ratios, marginal effects, differences in means) with confidence intervals, not just $p$-values.
\end{itemize}
