\section{Background and Rationale}\index{Sectioning!Sections}

India's states exhibit wide variation in demographic, social and economic indicators, shaped by geography, culture, history and policy. A comparative study of two mountainous states, Mizoram in the Northeast and Himachal Pradesh in the Western Himalayas, allows us to ask: how do demographic outcomes (e.g., infant mortality, maternal age), social infrastructure (education levels, sanitation) and economic development paths interplay across different regional contexts? The findings can help tailor policy interventions that recognise state specific dynamics rather than assume one size fits all.

\section{State Profiles}\index{Sectioning!Sections}

\subsection{Mizoram}\index{Sectioning!Subsections}

\begin{itemize}
    \item \textbf{Population density} in Mizoram is low ($\sim$52 persons/${km}^2$ approx) and its projected population growth is around the national average (0.9 \%) as of 2022-23.
    \item \textbf{Health indicator}: Mizoram reports one of the lowest infant mortality rates in India, for example, an IMR of $\sim$3 to 4 deaths per 1000 live births in recent years.
    \item \textbf{Education}: According to the 2011 Census, the literacy rate in Mizoram was 91.33 \%.
    \item \textbf{Economy and workforce}: As of 2022-23, the working population is concentrated in services (45.7 \%), agriculture (43.1 \%) and manufacturing (5.4 \%) in Mizoram. 
    \item \textbf{Infrastructure and geography}: Hilly terrain, shifting cultivation patterns and remoteness present both developmental strengths (close community networks) and challenges (access to health/sanitation).
\end{itemize}

\subsection{Himachal Pradesh}\index{Sectioning!Subsections}

\begin{itemize}
    \item \textbf{Demographic transition}: Himachal Pradesh has achieved relatively favourable human development outcomes compared to many Indian states. For example, its IMR is reported at ~17 deaths per 1000 live births around 2020. 
    \item \textbf{Education and literacy}: Himachal shows strong literacy and school infrastructure, though exact recent figures would need updating from Census/NFHS.
    \item \textbf{Economy and structure}: The economy shows a move away from agriculture (which contributes 13 \% of GSVA) towards services and industry, signalling a structural shift.
    \item \textbf{Topography and access}: The mountainous terrain, although offering tourism and hydro-resources, poses challenges in connectivity, sanitation supply and service delivery in remote areas.
\end{itemize}

\section{Key Thematic Areas of Study}
This report focuses on several thematic relationships across both states:
\begin{enumerate}
    \item \textbf{Infant mortality vs mother’s age}: e.g., how younger or older maternal ages correlate with infant outcomes.
    \item \textbf{Education level vs toilet facility types}: e.g., does higher educational attainment correspond to households having improved sanitation (water closet, piped sewer, etc)?
    \item \textbf{Other socio-economic/demographic linkages}: For instance, links between female labour-force participation, household amenities, and health outcomes; or comparisons across rural versus urban within each state.
\end{enumerate}

\section{Specific Data Snapshots}
Here are some concrete figures to anchor the analysis:
\begin{itemize}
    \item \textbf{Mizoram's IMR}: Rural-data shows ~3 per 1000 live births around 2020.
    \item \textbf{Himachal Pradesh's IMR}: Reported at ~17 per 1000 live births in 2020.
    \item \textbf{Mizoram's literacy rate}: 91.33 \% in 2011 Census.
    \item \textbf{Education-sanitation data}: Census table 'Availability and type of latrine facility -2001, 2011' covers latrine types by household and can be linked to education statistics. 
\end{itemize}

